\documentclass{article}
\usepackage{tikz}
\usepackage{ifthen}
\usetikzlibrary{calc}

\makeatletter
\newcommand{\gettikzy}[2]{
  \tikz@scan@one@point\pgfutil@firstofone#1\relax
  \edef#2{\the\pgf@y}
}
\makeatother

\newcommand{\blockinput}[2]{
  \draw #1 -- #2;
  \draw[fill] #2 circle [radius=2pt];
}

\newcommand{\blockoutput}[2]{
  \draw #1 -- #2;
  \draw let
    \p1 = #1,
    \p2 = #2
  in
    \ifdim\y1=\y2
      ($ #1 + (0,.2) $) -- ($ #1 - (0,.2) $)
    \else
      ($ #1 + (.2,0) $) -- ($ #1 - (.2,0) $)
    \fi ;
}

\newcommand{\parser}[1]{
  \draw #1 rectangle ($ #1 + (5,-4) $);
  \blockinput{($ #1 + (-1,-1) $)}{($ #1 + (.5,-1) $)};
  \blockinput{($ #1 + (-1,-3) $)}{($ #1 + (.5,-3) $)};
  \blockoutput{($ #1 + (4.5,-2) $)}{($ #1 + (6,-2) $)};
}

\begin{document}
\begin{tikzpicture}[scale=0.5]
% lhs
\parser{(3,7)};
% |-->
\draw [|->] (10,5) -- (14,5);
% rhs
\parser{(20,7)};
\blockinput{(15,6)}{(20.5,6)};
% ref selector
\draw (17,3) rectangle (19,5);
\blockinput{(15,4)}{(17.5,4)};
\blockinput{(18,2)}{(18,3.5)};
\blockoutput{(18.5,4)}{(20.5,4)};
% ref duplicator
\draw (26,4) rectangle (28,6);
\blockinput{(25,5)}{(26.6,5)};
\blockoutput{(27.5,5)}{(30,5)};
% outer
\draw (16,1) rectangle (29,9);

\end{tikzpicture}
\end{document}